\documentclass[a4paper,12pt]{article}
\usepackage{booktabs} 
\usepackage{array}
\usepackage{longtable}
\usepackage[utf8]{inputenc}
\usepackage{graphicx}
\usepackage{geometry}
\geometry{margin=1in}
\usepackage{setspace}
\usepackage{titling}


\pagenumbering{gobble}

\begin{document}

\begin{titlepage}
    \centering
    
    \includegraphics[width=0.25\textwidth]{images/logo.png}\par\vspace{1cm}
    
    {\scshape\LARGE Politecnico di Torino \par}
    \vspace{0.5cm}
    \vspace{2cm}
    
    {\Large\bfseries Microelectronic Systems\par}
    \vspace{1.5cm}
    

    \begin{flushleft}
        \large
        \begin{tabbing}
        \hspace{6cm} \= \kill
        \textbf{Student:} Luca Genca \hspace{1.5cm} \= \textbf{Student:} Alessio Delli Colli \\
        \textbf{ID:} XXXXXXX \> \textbf{ID:} YYYYYYY \\
        \end{tabbing}
    \end{flushleft}
    
    \vfill

    \vfill
    {\large \today\par}
\end{titlepage}
\section{Intruduction}
The project consists in the creation of a dlx processor in vhdl.
The dlx is a RISC processor with a Harvard architecture, which means that it uses two different memories for data and instructions.
The processor is organized in a five-stage pipeline. At instruction fetch (IF) the control unit reads the current instruction pointed to by the PC; decode (DE) reads the register file and prepares operands and immediates; execute (EX) performs ALU operations; memory (MEM) performs data memory accesses; and writeback (WB) updates the register file.\\
It has 3 different types of instructions: 
\begin{table}[ht]
\centering
\begin{tabular}{|c|p{8cm}|}
\hline
\textbf{Instruction Type} & \textbf{Description and Example} \\
\hline
\textbf{R-type} & Instructions that operate on two source registers and store the result in a destination register. Example: \texttt{add \$r1, \$r2, \$r3} \\
\hline
\textbf{I-type} & Instructions that involve a register and an immediate value. Used for arithmetic with constants, memory access, and branching. Example: \texttt{addi \$r1, \$r2, 10} \\
\hline
\textbf{J-type} & Instructions perform unconditional jumps. Example: \texttt{j 40} \\
\hline
\end{tabular}
\caption{DLX Processor Instruction Types}
\end{table}\\
The complete implemented instruction set is the following:
\begin{table}[h!]
\centering
\caption{Implemented Instructions with Opcodes and Function Codes}
\begin{tabular}{|l|c|l|c|l|c|}
\hline
\textbf{I/J-type} & \textbf{Opcode} & \textbf{Int R-type} & \textbf{Func} & \textbf{Float R-type} & \textbf{Func} \\
\hline
j       & 0x02 & sll   & 0x02 & multf & 0x02 \\
jal     & 0x03 & srl   & 0x03 & divf  & 0x03 \\
beqz    & 0x04 & sra   & 0x07 &  &  \\
bnez    & 0x05 & add   & 0x20 &  &  \\
addi    & 0x08 & addu  & 0x21 &  &  \\
addui   & 0x09 & sub   & 0x22 &  &  \\
subi    & 0x0a & subu  & 0x23 &  &  \\
subui   & 0x0b & and   & 0x24 &  &  \\
andi    & 0x0c & or    & 0x25 &  &  \\
ori     & 0x0d & xor   & 0x26 &  &  \\
xori    & 0x0e & seq   & 0x28 &  &  \\
lhi     & 0x0f & sne   & 0x29 &  &  \\
jr      & 0x12 & slt   & 0x2a &  &  \\
jalr    & 0x13 & sgt   & 0x2b &  &  \\
slli    & 0x14 & sle   & 0x2c &  &  \\
nop     & 0x15 & sge   & 0x2d &  &  \\
srli    & 0x16 & mult  & 0x0e &  &  \\
srai    & 0x17 & div   & 0x0f &  &  \\
seqi    & 0x18 & sltu  & 0x3a &  &  \\
snei    & 0x19 & sgtu  & 0x3b &  &  \\
slti    & 0x1a & sgeu  & 0x3d &  &  \\
sgti    & 0x1b &       &      &  &  \\
slei    & 0x1c &       &      &  &  \\
sgei    & 0x1d &       &      &  &  \\
lb      & 0x20 &       &      &  &  \\
lbu     & 0x24 &       &      &  &  \\
lhu     & 0x25 &       &      &  &  \\
lw      & 0x23 &       &      &  &  \\
sb      & 0x28 &       &      &  &  \\
sw      & 0x2b &       &      &  &  \\
sltui   & 0x3a &       &      &  &  \\
sgtui   & 0x3b &       &      &  &  \\
sleui   & 0x3c &       &      &  &  \\
sgeui   & 0x3d &       &      &  &  \\
\hline
\end{tabular}
\end{table}
\clearpage
\section{Control Unit}
We used a hardwired control unit with a 23-bits control word
\begin{table}[ht]
  \centering
  \small
  \caption{Control signals and their meanings}
  \label{tab:control-signals}
  \begin{tabular}{@{}l p{0.72\textwidth}@{}}
    \toprule
    \textbf{Signal} & \textbf{Meaning} \\
    \midrule
    PC\_LATCH\_EN     & Enables updating of the program counter (PC) register. \\
    RegA\_LATCH\_EN   & Enables the register containing the first operand in I and R type instructions. \\
    RegB\_LATCH\_EN   & Enables the register containing the second operand in R type instructions. \\
    RegIMM\_LATCH\_EN & Enables the register containing the immediate value in I and J type instructions.. \\
    RFR1\_EN          & Enables read signal for register-file read port 1 (reads operand A). \\
    RFR2\_EN          & Enables read signal for register-file read port 2 (reads operand B). \\
    RF\_EN            & Global register-file enable. \\
    ALU\_OUTREG\_EN   & Enables saving of the ALU result into the ALU output register for the next stage. \\
    MUX\_B            & Selects signal for the ALU second operand multiplexer (e.g. choose between RegB / immediate). \\
    MUX\_A            & Selects signal for the ALU A second operand multiplexer (e.g. choose between RegA / PC). \\
    MEM\_LATCH\_EN    & Enables saving of address/data into the memory-stage pipeline register. \\
    EQ\_COND          & Equality condition (used for branching). \\
    JUMP\_EN          & Enables jumping (true for both jumps and branches). \\
    JUMP              & Indicates that a jump needs to be performed instead of a branch. \\
    BYTE              & Select byte-sized memory access (8-bit load/store). \\
    DRAM\_WE          & DRAM write enable. \\
    LMD\_LATCH\_EN    & Enables the memory register. \\
    SEL\_MEM\_ALU     & Select between memory data and ALU result for the write-back. \\
    RF\_WE            & Register-file write enable. \\
    JAL               & needed for the JAL instruction. \\
    HALF\_WORD        & Select half-word memory access (16-bit load/store). \\
    H\_L               & High/Low half/byte selector (chooses high-half vs low-half or upper vs lower byte when writing). \\
    S\_U              & Sign/Unsigned control for load and store: chooses between sign-extension (signed) or zero-extension (unsigned). \\
    \bottomrule
  \end{tabular}
\end{table}
\section{Datapath}
The datapath forms the structural backbone of the processor. Its components are:
\begin{itemize}
    \item \textbf{Branch Target Buffer (BTB):} saves recently encountered branches and respective targets, enabling the program counter to jump immediately when seeing a branch or jump, thus reducing stalls.
    
    \item \textbf{Register File:} an array of registers with two asynchronous read ports and one synchronous write port.
    
    \item \textbf{Arithmetic Logic Unit (ALU):} performs integer arithmetic, bitwise logic, shift, comparison operations, multiplications, and divisions.
    
    \item \textbf{Pipeline / Skew Registers:} small registers inserted between pipeline stages or to skew signals and align timing. These registers are also used for hazard detection.
\end{itemize}
\subsection{Branch Target Buffer (BTB)}
The BTB follows a simple prediction algorithm based on the last observed behavior of each branch. Whenever a branch or jump instruction is executed, the BTB entry corresponding to its program counter (PC) is updated with the new outcome — either \textit{taken} or \textit{not taken}. On subsequent encounters of the same instruction, the processor consults the BTB and uses the most recent prediction to decide whether to fetch from the branch target address or continue sequentially.
This one-bit prediction strategy may mispredict alternating branches, but it performs well for branches with stable behavior, such as loops.
\subsection{ALU}
The ALU is composed of several components: adder, comparator, pipelined multiplier, shifter and divider.
\subsubsection{adder}
The adder is based on the adder from the Pentium4.
\subsubsection{comparator}
\subsubsection{multiplier}
\subsubsection{divider}
\section{Physical Synthesis}
\subsection{Synthesis}
\subsection{Place and route}
\end{document}
