\documentclass[a4paper,12pt]{article}
\usepackage{booktabs} 
\usepackage{array}

\usepackage[utf8]{inputenc}
\usepackage{graphicx}
\usepackage{geometry}
\geometry{margin=1in}
\usepackage{setspace}
\usepackage{titling}


\pagenumbering{gobble}

\begin{document}

\begin{titlepage}
    \centering
    
    \includegraphics[width=0.25\textwidth]{images/logo.png}\par\vspace{1cm}
    
    {\scshape\LARGE Politecnico di Torino \par}
    \vspace{0.5cm}
    \vspace{2cm}
    
    {\Large\bfseries Microelectronic Systems\par}
    \vspace{1.5cm}
    

    \begin{flushleft}
        \large
        \begin{tabbing}
        \hspace{6cm} \= \kill
        \textbf{Student:} Luca Genca \hspace{1.5cm} \= \textbf{Student:} Alessio Delli Colli \\
        \textbf{ID:} XXXXXXX \> \textbf{ID:} YYYYYYY \\
        \end{tabbing}
    \end{flushleft}
    
    \vfill

    \vfill
    {\large \today\par}
\end{titlepage}
\section{Intruduction}
The project consists in the creation of a dlx processor in vhdl.
The dlx is a RISC processor with a Harvard architecture, which means that it uses two different memories for data and instructions.
The processor is organized in a five-stage pipeline. At instruction fetch (IF) the control unit reads the current instruction pointed to by the PC; decode (DE) reads the register file and prepares operands and immediates; execute (EX) performs ALU operations; memory (MEM) performs data memory accesses; and writeback (WB) updates the register file.\\

\section{Control Unit}
We used a hardwired control unit with a 23-bits control word
\begin{table}[ht]
  \centering
  \small
  \caption{Control signals and their meanings}
  \label{tab:control-signals}
  \begin{tabular}{@{}l p{0.72\textwidth}@{}}
    \toprule
    \textbf{Signal} & \textbf{Meaning} \\
    \midrule
    PC\_LATCH\_EN     & Enables updating of the program counter (PC) register. \\
    RegA\_LATCH\_EN   & Enables the register containing the first operand in I and R type instructions. \\
    RegB\_LATCH\_EN   & Enables the register containing the second operand in R type instructions. \\
    RegIMM\_LATCH\_EN & Enables the register containing the immediate value in I and J type instructions.. \\
    RFR1\_EN          & Enables read signal for register-file read port 1 (reads operand A). \\
    RFR2\_EN          & Enables read signal for register-file read port 2 (reads operand B). \\
    RF\_EN            & Global register-file enable. \\
    ALU\_OUTREG\_EN   & Enables saving of the ALU result into the ALU output register for the next stage. \\
    MUX\_B            & Selects signal for the ALU second operand multiplexer (e.g. choose between RegB / immediate). \\
    MUX\_A            & Selects signal for the ALU A second operand multiplexer (e.g. choose between RegA / PC). \\
    MEM\_LATCH\_EN    & Enables saving of address/data into the memory-stage pipeline register. \\
    EQ\_COND          & Equality condition (used for branching). \\
    JUMP\_EN          & Enables jumping (true for both jumps and branches). \\
    JUMP              & Indicates that a jump needs to be performed instead of a branch. \\
    BYTE              & Select byte-sized memory access (8-bit load/store). \\
    DRAM\_WE          & DRAM write enable. \\
    LMD\_LATCH\_EN    & Enables the memory register. \\
    SEL\_MEM\_ALU     & Select between memory data and ALU result for the write-back. \\
    RF\_WE            & Register-file write enable. \\
    JAL               & needed for the JAL instruction. \\
    HALF\_WORD        & Select half-word memory access (16-bit load/store). \\
    H\_L               & High/Low half/byte selector (chooses high-half vs low-half or upper vs lower byte when writing). \\
    S\_U              & Sign/Unsigned control for load and store: chooses between sign-extension (signed) or zero-extension (unsigned). \\
    \bottomrule
  \end{tabular}
\end{table}
\end{document}
